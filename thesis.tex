\documentclass[11pt,edeposit,draftthesis]{uiucthesis2020}

% {{{ packages

\usepackage{blindtext}
\usepackage{cite}
\usepackage{array}
\usepackage{booktabs}
% math
\usepackage{fixmath}
\usepackage{amsmath}
\usepackage{amsthm}
\usepackage{amssymb}
\usepackage{stmaryrd}

% pretty links
\usepackage{xparse}
\usepackage{hyperref}
\usepackage{cleveref}
\usepackage{graphicx}
\hypersetup{
    colorlinks=true,
    urlcolor=blue,
    citecolor=black,
    linkcolor=black
}

% better environments
\usepackage[shortlabels]{enumitem}
\usepackage{booktabs}

% fancier font
\usepackage[sc]{mathpazo}
% better typography
\usepackage[activate={true,nocompatibility}, % activate protrusion and font expansion
            final,              % enable microtype, use draft to disable
            tracking=true,
            kerning=true,       % optimise interactions between characters
            spacing=true,       % more uniform spacing between words
            factor=1100,        % more protrusion
            stretch=10,         % smaller values (default 20, 20) to avoid blurring
            shrink=10]{microtype}
\microtypecontext{spacing=nonfrench}
\SetTracking{encoding={*}, shape=sc}{40}

% }}}

% {{{ commands

\NewDocumentCommand \dx { O{x} } {\,\mathrm{d} #1}
\NewDocumentCommand \vect { m } { \mathbold{#1} }
\NewDocumentCommand \jump { m } { \left\llbracket #1 \right\rrbracket }
\NewDocumentCommand \avg { m } { \left\langle #1 \right\rangle}
\NewDocumentCommand \od { m m } { \dfrac{\mathrm{d} #1}{\mathrm{d} #2} }
\NewDocumentCommand \pd { m m } { \dfrac{\partial #1}{\partial #2} }

% }}}

% {{{ title

\title{Magnetic ordering and spin wave dynamics in transition metal arsenides}
\author{Manohar H. Karigerasi}
\department{Materials Science and Engineering}

\schools{
B. Tech., Indian Institute of Technology Roorkee, 2016 \\
M. S., University of Illinois at Urbana-Champaign, 2017
}

\phdthesis
\advisor{Daniel P. Shoemaker}
\degreeyear{2021}

\committee{
Associate Professor Daniel P. Shoemaker, Chair \\
Professor David G. Cahill \\
Professor Jian-Min Zuo \\
Research Assistant Professor Gregory MacDougall \\
}

% }}}

\begin{document}

\maketitle

% {{{ front matter

\begin{frontmatter}

\begin{abstract}
Metallic antiferromagnets have gained interest in recent times due to the possibility of being useful as a memory device. Arsenic forms a large pool of magnetic metals in combination with other transition metals that have largely been ignored so far. In this report, we discover a new ternary metallic arsenide in the Cu-Mn-As phase space, identify its chemical and magnetic structure, and characterize its electrical and magnetic properties. We also carry out the magnetic structure refinement of Mn$_3$As$_2$ from neutron powder diffraction data at different temperatures to understand the magnetic ordering in Mn-As compounds. Using inelastic neutron scattering measurements, we determine exchange interactions in Fe$_2$As, which has the same structure as CuMnAs, showing a highly 2D magnon character although the phonons are 3D. Finally, we report a magnetic-structural coupled transition across 300 K in tetragonal CuMnAs and determine the correct magnetic structure of the compound.
\end{abstract}

\chapter*{Acknowledgments}

This project would not be possible without many people. Firstly, thanks to my advisor, Prof. Daniel P. Shoemaker.

%\begin{dedication}
%To Coffee
%\end{dedication}

\tableofcontents
\listoftables
\listoffigures

\end{frontmatter}

% }}}

% {{{ main matter

\begin{mainmatter}

%##################chapter 1
\chapter{Introduction}

\section{Magnetic information storage}

% Information 
In information memory storage devices, there is typically a trade-off between the optimum speed or response time and the complexity and size of memory storage \cite{Wing1986}. Volatile memory refers to temporary memory storage where the data is lost when the power is removed. Volatile memory such as SRAM (static random access memory) and DRAM (dynamic random access memory) are used as CPU caches and main memory respectively. SRAM has much faster access times and does not require periodic refreshing. However, it requires four to six transistors per bit as compared to one transistor and one capacitor in DRAM devices \cite{Meena2014}. Non-volatile memory (NVM) storage devices on the other hand, retain their data for a long period of time until perturbed. Modern computers mostly use flash memory based solid state drives (SSD) and magnetic hard disk drives (HDD) for storing large amounts of data permanently \cite{Meena2014}. The first HDD was invented in 1956 by IBM and since then, has seen more than eight orders of magnitude improvement in the storage density. However, the trilemma in magnetic recording between poor thermal stability, coercive fields and signal-to-noise ratio has resulted in the HDD reaching a saturation limit in their device performance \cite{Krishnan2016}. Flash memory uses floating gate MOSFETs (metal oxide semiconductor field effect transistors) to store memory and does not contain any moving parts. Although SSD have dominated the NVM marketshare last few years, there is an increasing need for alternative NVM technologies that are fast, low power consuming and have high storage density \cite{Meena2014}.

% Emerging NVM
One such emerging NVM is MRAM (magnetoresistive random access memory). Unlike flash memory which uses electronic charge as a medium of memory storage, MRAM uses the electronic spin degree of freedom to store information. Unlike charge based storage devices, MRAM is stable against perturbations such as ionizing radiation \cite{Wadley2016}. MRAM devices consist of cells with magnetic tunnel junctions (MTJ) that have two ferromagnet (FM) layers separated by an insulating layer. One of the layer is pinned where the magnetization orientation is fixed and acts as a reference layer. Depending on the orientation of the free layer, the tunneling magnetoresistance (TMR) is high or low and hence, memory can be read out using electrical currents \cite{Krishnan2016}. Early MRAMs were written by induced fields from heavy currents passed on the adjacent layer. With recent developments in spin transfer torque (STT) in ferromagnets, it has become possible to write using electrical currents \cite{Chappert2007}. This has reduced the power consumption significantly and made commercialization of MRAM devices possible \cite{Krishnan2016,Bhatti2017}.

\section{Antiferromagnets for potential applications as a memory unit}

% Gomonay 2010 paper
Historically, antiferromagnets (AFM) have been used as inactive components in MTJ, primarily in exchange biasing the pinned FM layer. However in 2010, Gomonay \emph{et al.} \cite{Gomonay2010} proposed electrical switching of AFMs using STT by passing a spin polarized current injected from a fixed FM layer through the AFM layer. The electrical current gets spin polarized in the FM layer and transfers its angular momentum to the AFM moments to switch it from one orientation to another. There are advantages to using AFM over FM in MRAM devices. AFM are not easily affected by external magnetic fields and do not produce stray fields of their own. They have smaller domains which would allow for higher storage densities \cite{Wadley2016}. Since the precession frequency of AFM moments is determined the geometric mean of exchange and anisotropy energies, the dynamics in AFM materials occur at THz timescales which is two orders of magnitude higher than in FM \cite{Siddiqui2020}. Although the AFM can be switched using electrical currents from parallel to perpendicular orientation with respect to the FM magnetization direction, the reverse process cannot be obtained using electrical current. High magnetic fields above the spin flop transition of the AFM needs to be applied in order to switch back the AFM to its original state \cite{Gomonay2010}.

% Wadley 2016 paper
Unlike previously discussed STT MRAM devices, spin orbit torque based electrical switching in broken inversion symmetry FM does not require the presence of a FM polarizer \cite{Manchon2008}. This concept of the presence of relativistic fields is applicable to AFM as well provided the local moments do not sit on centrosymmetric sites. If the two sublattices are related to each other by a center of inversion, then the current induced spin polarized fields are staggered across the two sublattices \cite{Zelezny2014,Zelezny2017,Wadley2016}. This results in a uniform fieldlike torque experienced by the order parameter. This is possible in bulk materials that are globally centrosymmetric but locally non-centrosymmetric and the two sublattices are related to each other by a center of inversion. It was initially demonstrated in the case of epitaxially grown tetragonal CuMnAs thin films on GaP substrate \cite{Wadley2016} and since then, it has also been shown in Mn$_2$Au and CuMnAs sputtered films as well \cite{Meinert2018,Matalla-Wagner2019}. Observation of electrical switching behavior in AFM requires the presence of degenrate N\'eel vectors like in CuMnAs as shown in Fig. \ref{fig:tet-CuMnAs}(a) as opposed to compounds like MnF$_2$ where the Mn moments point along $c$ in Fig. \ref{fig:tet-CuMnAs}(b).


\begin{figure}
\centering\includegraphics[width=\columnwidth]{figures/ch1/CuMnAs-MnF2.png} \\
\caption{\label{fig:tet-CuMnAs}
The magnetic structures of tetragonal CuMnAs and MnF$_2$ are shown in (a) and (b), respectively.
}
\end{figure}

\section{Exploration of Cu-Mn-As phase space}

% Why are CuMnAs compounds popular?
Compounds in Cu-Mn-As phase space have attracted a lot of attention in recent times mainly due to the exotic properties of tetragonal and orthorhombic CuMnAs. As mentioned earlier, tetragonal CuMnAs was the first antiferromagnet where electrical switching was supposedly demonstrated. Orthorhombic CuMnAs, shown in Fig. \ref{fig:ort-CuMnAs}, was the first magnetic compound to be proposed as a Dirac semimetal. It is a special compound where the inversion and time reversal symmetry of the magnetic structure is broken but their combined symmetry ($PT$ symmetry) is still preserved. Based on the orientation of the AFM order parameter, the compound changes from a conducting to an insulating phase. Hence, there are voltage based switching applications that have been proposed for this compound \cite{Kim2018}.

\begin{figure}
\centering\includegraphics[width=0.5\columnwidth]{figures/ch1/ort-CuMnAs.png} \\
\caption{\label{fig:ort-CuMnAs}
Magnetic structure of orthorhombic CuMnAs
}
\end{figure}


\begin{figure}
\centering\includegraphics[width=0.8\columnwidth]{figures/ch1/Cu-Mn-As phase diagram.png} \\
\caption{\label{fig:Cu-Mn-As}
Cu-Mn-As ternary phase diagram highlighting some of the known ternary compounds in blue and known magnetic structures in red. Not all compounds in this system are shown here.
}
\end{figure}


% The dual problem
Despite the growing importance of the compounds in Cu-Mn-As system, the Cu-Mn-As ternary phase space has not been explored properly. There are four known ternary compounds including both the polymorphs of CuMnAs, orthorhombic CuMn$_3$As$_2$ and Cu$_2$Mn$_4$As$_3$ as shown in Fig. \ref{fig:Cu-Mn-As} \cite{Nateprov2011,MacA2012,Wadley2013,Uhlirova2015}. Bulk orthorhombic CuMnAs can be grown using traditional solid state synthesis routes by Cu, Mn and As powders in stoichiometric ratio and heating the powders to 1000$^\circ$C. In order to synthesize pure bulk tetragonal CuMnAs, we have to go off-stoichiometry and either substitute Mn with Cu or As \cite{Uhlirova2015}. Hence, it is important to explore different regions in the Cu-Mn-As system and verify the stability of different ternary compounds. The magnetic structures in the Cu-Mn-As system have also not been identified for most of the compounds. Apart from the four Cu-Mn-As ternary compounds, there are more than ten Mn-As binary compounds present in this system. The magnetic structures are known only for the two previously mentioned CuMnAs compounds, Mn$_2$As and MnAs as shown in Fig. \ref{fig:Cu-Mn-As} \cite{MacA2012,Wadley2013,Hills2015,Wadley2015,Austin1962,Bacon1955}. Since most, if not all, the binary Mn-As compounds are metallic, there is a need to magnetically characterize the compounds and identify their magnetic structures.

\section{Exchange interactions in Cu$_2$Sb type structures}

% Exchange interactions in Cu$_2$Sb materials.
If we want to understand the electrical switching behavior in metallic antiferromagnets, we should be able to quantify the fundamental energies such as magnetocrystalline anisotropy and exchange interactions in materials like CuMnAs. CuMnAs has a Cu$_2$Sb structure type. Other materials with this structure includes Mn$_2$As, Cr$_2$As, Fe$_2$As, CrMnAs, MnFeAs etc. \cite{Lutz-Kappelman2018,Zhang2013,Zhang2015}. Although Mn$_2$As, Cr$_2$As and Fe$_2$As have the same structure, the magnetic ground state is different in all three compounds. The strength and sign of direct exchange interactions between two magnetic atoms is a result of the nature of orbital overlap between the two magnetic atoms \cite{Zhang2013}. Since these materials are metallic, there are two contributions to indirect exchange interactions. One contribution arises from superexchange interactions mediated by As atoms and the other contribution comes from RKKY (Ruderman–Kittel–Kasuya–Yosida) interactions \cite{Zhang2015}. It is important that we are able to determine what spin interactions are relevant and how does it affect the magnetic ordering in these materials. It is also crucial that we are able to verify the computational methods and the exchange coupling values obtained from these methods so that we can use these methods for other systems as well.

%##################chapter 2
\chapter{Theory of electrical switching in metallic antiferromagnets}

%Introduction
Electrical switching in any magnetic compound is a series of events involving current induced spin polarization (CISP) of charge carriers and different components of CISP exerting different torque on the magnetic moments of the atoms. The nature of CISP is set by the crystal symmetry. In previous studies of CuMnAs and Mn$_2$Au, it was stated that the compound should globally centrosymmetric but locally non-centrosymmetric and the two sublattices should be related to each other by a center of inversion \cite{Zelezny2014,Zelezny2017,Wadley2016}. Is this a necessary condition for observing a staggered spin polarization configuration and can it be applied to general cases? These are some of the questions we will answer in this chapter. Once we have determined the required symmetry criteria, we will filter out metallic antiferromagnetic candidates from large databases of materials such as MPDS (Materials Platform for Data Science) and ASM (ASM International), and analyze the effect of CISP on the torque experienced by the order parameter.

\section{Hidden spin polarization in centrosymmetric crystals}

% Explanation of R2-D2 effect
It has been known for quite some time that in materials (even non-magnetic) having large spin orbit coupling (SOC) and lacking a center of inversion, magnetic fields are induced. When materials possess structural inversion asymmetry such as in quantum wells and other heterostructures, this effect is called the Rashba effect and it results in a helical type spin texture. When this occurs in materials that lack bulk inversion symmetry, the effect is called the Dresselhaus effect and it results in a unique spin texture \cite{Zhang2014}. Zhang \emph{et al.} \cite{Zhang2014} argued that since SOC is a relativistic effect, instead of considering the symmetry of the entire unit cell, one should check for atomic site symmetry to understand SOC induced spin polarization. Based on this argument, there are four cases possible as shown in Table \ref{tab:RD_effect}.

R1 and D1 effects refer to conventional Rashba and Dresselhaus spin polarization respectively. In materials that are globally centrosymmetric, hidden spin polarization is possible. There is local spin polarization near non-centrosymmetric sites but when summed over the entire unit cell, the net spin polarization is zero. This effect is called the R2 or D2 effect corresponding to Rashba or Dresselhaus effect, respectively in centrosymmetric crystals. The total spin polarization of the unit cell is the vector sum of all the local spin polarizations in the unit cell. Non-centrosymmetric point groups can be further divided into polar and non-polar groups. Local Rashba effect requires the presence of polar point groups on atomic sites and the local Dresselhaus effect requires the presence of non-polar point groups at the atom sites. The presence of local spin polarization in centrosymmetric crystals opens the avenue for studying spin polarization in a larger group of compounds including metallic antiferromagnets.

\renewcommand{\arraystretch}{1.2}
\begin{table}
\caption{\label{tab:RD_effect} 
Different cases of spin polarization depending on the symmetry of atomic sites and the unit cell \cite{Zhang2014}.}
\centering
\begin{tabular}{>{\raggedright\arraybackslash}p{5cm}>{\raggedright\arraybackslash}p{3cm}>{\raggedright\arraybackslash}p{3cm}>{\raggedright\arraybackslash}p{3cm}}
\hline\hline
\textbf{} & \textbf{All non-polar point groups} & \textbf{At least one polar point groups} & \textbf{All centrosymmetric point groups}\\
\hline
\textbf{Non-Centrosymmetric space group} & D1 effect & R1/D1 effect & Not possible\\
\hline
\textbf{Centrosymmetric space group} & D2 effect & R2/D2 effect & No spin polarization\\
\hline\hline
\end{tabular}
~\\
\end{table}

\section{Finding metallic antiferromagnetic candidates with R2-D2 effect}

% Replace this figure maybe
\begin{figure}
\centering\includegraphics[width=\columnwidth]{figures/ch2/wadley_1.png} \\
\caption{\label{fig:wadley_1}
The schematic of the Fermi surface of the local sectors near Mn1 and Mn2 atoms are shown in (a) and (b), respectively. The magnetic structure of tetragonal CuMnAs is shown in (c) which also includes the highlighted local sectors.
}
\end{figure}

% Explain use of Zunger paper in CuMnAs

In spin orbit torque based switching of metallic antiferromagnets, the same arguments apply as before except that we only care about the point group symmetry at the magnetic atom sites. For example, the magnetic structure of CuMnAs is shown in Fig. \ref{fig:wadley_1}(c). Cu and As atoms have no moments and sit on D$_{2d}$ and C$_{4v}$ sites, respectively. Since only the Mn atoms have moments, we must take the site symmetry of Mn atoms into consideration. Mn atom sites have C$_{4v}$ point group symmetry which is polar. As seen in Fig. \ref{fig:wadley_1}(a,b), the schematic of the Fermi surface corresponding to the Mn layer has a Rashba-like spin texture. When the current is applied, through inverse spin galvanic effect, there is a spin polarization of the conduction electrons near Mn sites. The spin polarization is staggered across the two Mn sublattices and through exchange coupling, a torque is experienced by the Mn moments. From the example of CuMnAs above, we can search for compounds from databases that are antiferromagnets or are likely to be antiferromagnets where the magnetic atoms sit on a polar point group. Such a search can be made for compounds which are not complicated by the presence of many magnetic sites with different point groups. The general case applicable for all magnetic compounds will be considered in the following sections. Following is the general procedure for extracting potential metallic antiferromagnet candidates from databases -

% Procedure list
\begin{enumerate}
\item Search for known metallic antiferromagnets from database having tetragonal, trigonal or hexagonal crystal system.
\item Remove duplicate and non-centrosymmetric compounds.
\item Identify the Néel temperature and magnetic ordering.
\item Filter out compounds having Néel temperature $>$ 300 K.
\item Identify the nature of atomic site symmetry and select compounds.
\item Study available literature to see if any compounds can be synthesized as single crystals.
\end{enumerate}

This procedure assumes that we are starting with a collection of known or possible metallic antiferromagnets. This is true in case of compounds downloaded from MPDS. In case of ASM database, the compounds are metallic but their magnetism is not known beforehand. We will have to make some assumptions based on the chemical nature and stoichiometry of the ions present in the compound in order to select possible antiferromagnets. Tetragonal, trigonal or hexagonal crystal systems are preferred since they allow for the presence of multiple degenerate axes for N\'eel vector orientation. It is still important to check if the magnetic structure for the compound is known or not and if the N\'eel vector points along the degerate axes. Carrying out the above mentioned steps for compounds in MPDS and ASM database gives us a list of compounds  in Table \ref{tab:metallic_afm_candidates} that have been grouped together based on their structure types.

\renewcommand{\arraystretch}{1.2}
\begin{table}
\caption{\label{tab:metallic_afm_candidates} 
List of metallic antiferromagnetic candidates filtered out from MPDS and ASM database and their metal atom site symmetries}
\centering
\begin{tabular}{>{\raggedright\arraybackslash}p{3cm}>{\raggedright\arraybackslash}p{4cm}>{\centering}p{3cm}>{\raggedright\arraybackslash}p{4cm}}
\hline\hline
\textbf{Structure type} & \textbf{List of compounds} & \textbf{Point group} & \textbf{Magnetism}\\
\hline
MoSi$_2$ & Mn$_2$Au, Cr$_2$Al, Ni$_2$Ta & C$_{4v}$ & Mn$_2$Au is a known candidate\\
\hline
HfFe$_6$Ge$_6$ & ScFe$_6$Ge$_6$ & C$_{2v}$ & AFM with N\'eel vector along $c$\\
\hline
Mg$_3$Cd & Mn$_3$Ga, Mn$_3$Ge, Mn$_3$Sn, Fe$_3$Ga, Co$_3$Mo, Co$_3$W, Ni$_3$In, Ni$_3$Sn, Ni$_3$Zr & C$_{2v}$ & Non-collinear AFM\\
\hline
IrIn$_3$ & CoGa$_3$, CoIn$_3$, FeGa$_3$ & C$_{2v}$ & FeGa$_3$ and CoGa$_3$ are diamagnets\\
\hline
MoNi$_4$ & MoNi$_4$, WNi$_4$ & C$_{1v}$ & Unknown\\
\hline
Ni$_2$Al$_3$ & Ni$_2$Al$_3$, Ni$_2$Ga$_3$, Ni$_2$In$_3$ & C$_{3v}$ & Unknown\\
\hline\hline
\end{tabular}
~\\
\end{table}

%Discuss the list of compounds obtained

The first group of compounds consist of materials in MoSi$_2$ structure type. Mn$_2$Au is a well-known compound that has been extensively studied for electrical switching applications \cite{Meinert2018,Bodnar2019}. Cr$_2$Al powders can be prepared using traditional solid state synthesis routes by heating to above 850$^\circ$C \cite{Susner2015}. Early neutron diffraction experiments suggest Cr moments align at 65$^\circ$ to the $ab$ plane \cite{Atoji1965}. However, a later article indicates that the determined magnetic structure may not be correct \cite{Kallel1967}. Regardless of whether the moments align in the $ab$ plane or not, Cr$_2$Al is an interesting candidate to study electrical switching behavior. ScFe$_6$Ge$_6$ is AFM at room temperature \cite{Venturini1992}. However, the Fe moments align along $c$ and does not satify our criteria \cite{Mazet2000}. Compounds in the Mg$_3$Cd structure type contain Kagome lattice of magnetic atoms. Mn$_3$Sn and Mn$_3$Ge are known to have a non-collinear spin arrangement. They have become popular recently for showing large anomalous hall and spin hall effect behavior \cite{Nakatsuji2015,Kubler2014,Kimata2019}. Electrical switching was also demonstrated in Mn$_3$Sn recently. In the fourth group of compounds, both FeGa$_3$ and CoGa$_3$ are known to show diamagnetic properties and hence they can be discarded \cite{Haussermann2002,Viklund2002,Zhang2017}. The magnetism is unknown in the final two groups of compounds in MoNi$_4$ and Ni$_2$Al$_3$ structure types and most of these compounds can be synthesized by the arc melting process \cite{Harker1944,Taylor1972}.

\section{Components of torque from non-equilibrium CISP}

% Zelezny 2017 paper

The previous section dealt with relatively simple magnetic compounds that had only one magnetic atom site and we were concerned with the point group of the site to determine whether the inverse spin galvanic effect would be observed or not. In the simplest sense, antidamping-like STT is induced by spin hall effect (SHE) and fieldlike SOT is generated from inverse spin galvanic effect. However, incomplete absorption of the spin current from SHE by the FM layer may produce a fieldlike torque, and spin relaxation and damping may induce an antidamping like component to the SOT \cite{Zelezny2017}. Using Kubo linear formalism, we can write CISP $\delta S_a = \chi_a E$ where E is the applied electric field and $\chi_a$ is the linear response tensor for the sublattice a. $\chi_a$ can be further divided into three components:

\begin{equation}
\chi_a = \chi_a^I + \chi_a^{II(a)} + \chi_a^{II(b)}
\end{equation}

where $\chi_a^I$ is the intraband term, and $\chi_a^{II(a)}$ and $\chi_a^{II(b)}$ are the interband terms. $\chi_a$ can also be broken down into even and odd terms:

\begin{equation}
\chi_a = \chi_a^{even} + \chi_a^{odd}
\end{equation}

where $\chi_a^{even} = (\chi_a([M])+\chi_a([-M]))/2$ and $\chi_a^{odd} = (\chi_a([M])-\chi_a([-M]))/2$. From the symmetry of the operators and the matrix element, it follows that -

\begin{equation}
\chi_a^{even} = \chi_a^{I} + \chi_a^{II(b)}
\end{equation}
\begin{equation}
\chi_a^{odd} = \chi_a^{II(a)}
\end{equation}

We assume that the system only has a weak disorder and hence, we can neglect $\chi_a^{II(b)}$. $\chi_a$ also depends on the direction of the magnetic moments where \^n is the direction of the N\'eel vector.

\begin{equation}
\chi_{a,i,j}(\hat{n}) = \chi_{a,i,j}^{(0)} + \chi_{a,i,j,k}^{(1)}\hat{n}_k + \chi_{a,i,j,k,l}^{(2)}\hat{n}_k\hat{n}_l + ...
\end{equation}

where the sum of the first term and every alternate term after that corresponds to $\chi_a^{even}$ and the sum of the remaining terms correspond to $\chi_a^{odd}$. $\chi_a^{(0)}$ is usually dominant, independent of magnetization and contributes to field-like torque. $\chi_a^{(1)}$ contributes to anti-damping like torque and $\chi_a^{(2)}$ can be neglected if $\chi_a^{(0)}$ is not 0 \cite{Zelezny2017}. The zeroth order term which contributes to a fieldlike torque consists of three components in the form of generalized Rashba and generalized Dresselhaus terms and a term that is proportional to the electric field as shown by the following equations:

\begin{equation}
\text{Generalized Rashba } \chi_a^{gR} = \begin{bmatrix} \chi_{11} & -\chi_{21} & 0 \\ \chi_{21} & \chi_{11} & 0 \\ 0 & 0 & 0 \end{bmatrix}
\end{equation}

\begin{equation}
\text{Generalized Dresselhaus } \chi_a^{gD} = \begin{bmatrix} \chi_{11} & \chi_{21} & 0 \\ \chi_{21} & -\chi_{11} & 0 \\ 0 & 0 & 0 \end{bmatrix}
\end{equation}

\begin{equation}
\text{Proportional to Electric field } \chi_a^{E} = \begin{bmatrix} \chi_{11} & 0 & 0 \\ 0 & \chi_{11} & 0 \\ 0 & 0 & \chi_{11} \end{bmatrix}
\end{equation}



Similarly, the first order term can also be broken into three different components \cite{Zelezny2017}. The summary of the linear response tensor components is also provided in the Table \ref{tab:chi_summary}

\renewcommand{\arraystretch}{1.3}
\begin{table}
\caption{\label{tab:chi_summary} 
Summary of the linear response tensor components for the CISP observable and electric field}
\centering
\begin{tabular}{>{\raggedright\arraybackslash}p{3.5cm}>{\raggedright\arraybackslash}p{3.5cm}>{\raggedright\arraybackslash}p{3.5cm}>{\raggedright\arraybackslash}p{3.5cm}}
\hline\hline
 & \boldmath{$\chi_a^I$} & \boldmath{$\chi_a^{II(a)}$} & \boldmath{$\chi_a^{II(b)}$}\\
\hline
\textbf{Component} &  Intraband & Interband (imaginary) & Interband (real)\\
\hline
\textbf{Disorder $\Gamma \rightarrow 0$} &  Diverges & Constant & Zero\\
\hline
\textbf{Cause} & Non-equilibrium Fermi-Dirac distribution & Intrinsic change in carrier wave function & Change in carrier wave function due to disorder/defects\\
\hline
\textbf{Alternate name} & $\chi_a^{even}$ or $\chi_a^{(0)}$ & $\chi_a^{odd}$ or $\chi_a^{(1)}$ & \\
\hline
\textbf{Dependence on magnetization} & Independent & Dependent & \\
\hline
\textbf{Torque} & Field-like & Anti-damping-like & \\
\hline\hline
\end{tabular}
~\\
\end{table}

\section{Spin polarization in CuMnAs, Mn$_2$Au and Fe$_2$As}

% Provide examples from code.

Zelezny \emph{et al.} \cite{Zelezny2017} provides a python code $symmetr$ for analyzing the linear response tensor for a number of observables such as current, spin, torque, position, spin current etc. and electric field. Using this code, we can test the reponse tensor of known compounds such as CuMnAs and Mn$_2$Au and also check for unknown compounds such as Fe$_2$As. Table \ref{tab:CISP_CuMnAs} shows the linear response tensor in case of CuMnAs or Mn$_2$Au. Both compounds have same spin polarization since they are both globally centrosymmetric and Mn atoms sit on C$_{4v}$ point group sites. Let us first understand $\chi^{even}$ for all the different cases presented here. Magnetism can be turned off to avoid including magnetization-based effects to the reponse tensor. Regardless of whether we toggle magnetism in Mn atoms or not, $\chi^{even}$ is 0 for the whole unit cell since it is centrosymmetric. When Mn magnetic moment is set to 0, $\chi^{even}$ for each sublattice corresponds to the response in conventional Rashba spin texture. Regardless of magnetism, the spin polarization in one site is opposite to that of another site as seen in the projected cases. When magnetism is turned on, $|\chi_{10}| = |\chi_{01}|$ is not valid anymore since we also have to consider higher terms in $\chi^{even}$. In case of $\chi^{odd}$, it is 0 when magnetization is turned off as also shown in Table \ref{tab:chi_summary}. When magnetic moments are allowed, there is spin polarization along $c$ direction.

\begin{table}
\caption{\label{tab:CISP_CuMnAs} 
Linear response tensor in CuMnAs and Mn$_2$Au assuming Mn moment to be 1 $\mu_B$.}
\centering
\begin{tabular}{>{\raggedright\arraybackslash}p{7cm}>{\centering\arraybackslash}p{3.5cm}>{\centering\arraybackslash}p{3.5cm}}
\hline\hline
\addlinespace[1.5ex]
 & \boldmath{$\chi^{even}$} & \boldmath{$\chi^{odd}$} \\
\addlinespace[1.5ex]
\hline
\addlinespace[1.5ex]
\textbf{Magnetic Mn. For the entire unit cell} & $\begin{bmatrix} 0 & 0 & 0\\ 0 & 0 & 0\\  0 & 0 & 0\\ \end{bmatrix}$ & $\begin{bmatrix} 0 & 0 & \chi_{02}\\ 0 & 0 & 0\\  \chi_{20} & 0 & 0 \end{bmatrix}$ \\
\addlinespace[1.5ex]
\hline
\addlinespace[1.5ex]
\textbf{Magnetic Mn. For each Mn sublattice} &  $\begin{bmatrix} 0 & \chi_{01} & 0\\ \chi_{10} & 0 & 0\\  0 & 0 & 0\\ \end{bmatrix}$ & $\begin{bmatrix} 0 & 0 & \chi_{02}\\ 0 & 0 & 0\\  \chi_{20} & 0 & 0\end{bmatrix}$\\
\addlinespace[1.5ex]
\hline
\addlinespace[1.5ex]
\textbf{Magnetic Mn. For one Mn sublattice projected onto another} & $\begin{bmatrix} 0 & -\chi_{01} & 0\\  -\chi_{10} & 0 & 0\\  0 & 0 & 0\\ \end{bmatrix}$ & $\begin{bmatrix} 0 & 0 & \chi_{02}\\ 0 & 0 & 0\\  \chi_{20} & 0 & 0\end{bmatrix}$\\
\addlinespace[1.5ex]
\hline
\addlinespace[1.5ex]
\textbf{Non-magnetic Mn. For the entire unit cell} & $\begin{bmatrix} 0 & 0 & 0\\ 0 & 0 & 0\\  0 & 0 & 0\\ \end{bmatrix}$ & $\begin{bmatrix} 0 & 0 & 0\\ 0 & 0 & 0\\ 0 & 0 & 0\\ \end{bmatrix}$\\
\addlinespace[1.5ex]
\hline
\addlinespace[1.5ex]
\textbf{Non-magnetic Mn. For each Mn sublattice} & $\begin{bmatrix} 0 & -\chi_{10} & 0\\ \chi_{10} & 0 & 0\\  0 & 0 & 0\\ \end{bmatrix}$ & $\begin{bmatrix} 0 & 0 & 0\\ 0 & 0 & 0\\ 0 & 0 & 0\\ \end{bmatrix}$\\
\addlinespace[1.5ex]
\hline
\addlinespace[1.5ex]
\textbf{Non-magnetic Mn. For one Mn sublattice projected onto another} & $\begin{bmatrix} 0 & \chi_{10} & 0\\ -\chi_{10} & 0 & 0\\  0 & 0 & 0\\ \end{bmatrix}$ & $\begin{bmatrix} 0 & 0 & 0\\ 0 & 0 & 0\\ 0 & 0 & 0\\ \end{bmatrix}$\\
\addlinespace[1.5ex]
\hline\hline
\end{tabular}
~\\
\end{table}

Table \ref{tab:CISP_Fe2As} shows linear response tensor components in Fe$_2$As. Fe$_2$As is complicated by the presence of two different Fe sites. Fe1, Fe2, Fe5 and Fe6 atoms shown in Fig. \ref{fig:Fe2As} sit on D$_{2d}$ site whereas Fe3, Fe4, Fe7 and Fe8 atoms sit on C$_{4v}$ sites. The table only shows cases where Fe moments have not been considered. However, the results with finite Fe moments in case of $\chi^{even}$ can be easily inferred from this table by removing the equality between the magnitudes of $\chi_{ij}$ and $\chi_{ji}$. As expected, $\chi^{odd}$ is a zero matrix when magnetism is turned off. $\chi^{even}$ is 0 when the entire unit cell is considered which is expected since the unit cell is centrosymmetric like in the previous case. The sublattices Fe3 and Fe4 sitting on polar point groups show conventional Rashba spin polarization behavior. When $\chi^{even}$ at Fe3 is projected onto Fe4, we can see that the response is exactly opposite to Fe4. In case of sublattices Fe1 and Fe2 sitting on non-polar point groups, $\chi^{even}$ corresponds to generalized Dresselhaus polarization when $\chi_{00}$ and $\chi_{11}$ have been set to 0. As expected, the spin polarization would be opposite on the two sublattices. The nature of $\chi^{odd}$ when magnetism is included has not been discussed here.



\begin{table}
\caption{\label{tab:CISP_Fe2As} 
Linear response tensor in Fe$_2$As when Fe moments have been set to 0 $\mu_B$.}
\centering
\begin{tabular}{>{\raggedright\arraybackslash}p{7cm}>{\centering\arraybackslash}p{3.5cm}>{\centering\arraybackslash}p{3.5cm}}
\hline\hline
\addlinespace[1.5ex]
 & \boldmath{$\chi^{even}$} & \boldmath{$\chi^{odd}$} \\
\addlinespace[1.5ex]
\hline
\addlinespace[1.5ex]
\textbf{For the entire unit cell} & $\begin{bmatrix} 0 & 0 & 0\\ 0 & 0 & 0\\  0 & 0 & 0\\ \end{bmatrix}$ & $\begin{bmatrix} 0 & 0 & 0 \\ 0 & 0 & 0\\  0 & 0 & 0 \end{bmatrix}$ \\
\addlinespace[1.5ex]
\hline
\addlinespace[1.5ex]
\textbf{For Fe1 and Fe2} &  $\begin{bmatrix} 0 & \chi_{10} & 0\\ \chi_{10} & 0 & 0\\  0 & 0 & 0\\ \end{bmatrix}$ & $\begin{bmatrix} 0 & 0 & 0 \\ 0 & 0 & 0\\  0 & 0 & 0 \end{bmatrix}$\\
\addlinespace[1.5ex]
\hline
\addlinespace[1.5ex]
\textbf{For Fe3 and Fe4} & $\begin{bmatrix} 0 & -\chi_{10} & 0\\  \chi_{10} & 0 & 0\\  0 & 0 & 0\\ \end{bmatrix}$ & $\begin{bmatrix} 0 & 0 & 0 \\ 0 & 0 & 0\\  0 & 0 & 0 \end{bmatrix}$\\
\addlinespace[1.5ex]
\hline
\addlinespace[1.5ex]
\textbf{For Fe1 projected onto Fe2} & $\begin{bmatrix} 0 & -\chi_{10} & 0\\ -\chi_{10} & 0 & 0 \\  0 & 0 & 0\\ \end{bmatrix}$ & $\begin{bmatrix} 0 & 0 & 0\\ 0 & 0 & 0\\ 0 & 0 & 0\\ \end{bmatrix}$\\
\addlinespace[1.5ex]
\hline
\addlinespace[1.5ex]
\textbf{For Fe3 projected onto Fe4} & $\begin{bmatrix} 0 & \chi_{10} & 0\\ -\chi_{10} & 0 & 0\\  0 & 0 & 0\\ \end{bmatrix}$ & $\begin{bmatrix} 0 & 0 & 0\\ 0 & 0 & 0\\ 0 & 0 & 0\\ \end{bmatrix}$\\
\addlinespace[1.5ex]
\hline
\addlinespace[1.5ex]
\textbf{For Fe1 projected onto Fe3} & No relation & No relation \\
\addlinespace[1.5ex]
\hline\hline
\end{tabular}
~\\
\end{table}

\begin{figure}
\centering\includegraphics[width=0.3\columnwidth]{figures/ch2/Fe2As.png} \\
\caption{\label{fig:Fe2As}
Magnetic structure of Fe$_2$As along with the Fe site numbers corresponding to Table \ref{tab:CISP_Fe2As}.
}
\end{figure}

\section{Conclusions}

There can be local spin polarization present even in centrosymmetric crystals provided the unit cell contains non-centrosymmetric point groups. In CISP based switching, it is important that the magnetic ions sit on non-centrosymmetric sites. The presence of polar point group is required for Rashba spin polarization and non-polar point group for Dresselhaus spin polarization. We need to consider both antidamping-like torque and fieldlike torque when considering spin polarization. In broken inversion symmetry 2D AFM, it is the antidamping-like torque that can provide a mechanism for switching since the fieldlike torque component of CISP is uniform. In Fe$_2$As, CISP from ISGE in Fe3 and Fe4 sublattices is similar to Mn in CuMnAs. However, there is no relation between the even component of CISP in Fe1 or Fe2 and Fe3 or Fe4 sublattices. This allows for two possibilities when the current is along $[100]$. If the torque acts in the same direction for both sets of Fe atoms, then it would provide a possible pathway for switching. However, if the fieldlike torque act in the opposite direction for both sets of Fe moments, then a large current threshold may be required before the moments can be possibly switched.

\section{Acknowledgements}

% Acknowledge Scott and Carmen

I would like to acknowledge two REU students, Scott Berens and Carmen Paquette, for compiling the list of metallic antiferromagnets from MPDS and ASM databases respectively. It saved me a lot of time and I was able to analyze the symmetry requirements on a much smaller list of compounds.





%##################chapter 3
\chapter{Materials synthesis and characterization}

\section{Bulk materials synthesis}

%As : Arsenic Polycrystalline lump, 2-8mm (0.08-0.3 in.), Puratronic, 99.9999% (Metals basis), Alfa Aesar
%Mn : 99.98% metals basis
%Cu : Copper powder, spherical, -100+325 mesh, 99.9% (metals basis), Alfa Aesar
%Fe old : Iron powder, -200 mesh, 99+% (metals basis), Alfa aesar
%Fe new : Iron powder, spherical, >99.99% (metal basis), 99.5%, Alfa Aesar

% Traditional solid state synthesis technique and provide ref to figures

Bulk polycrystalline and single crystals of all the samples are prepared by traditional solid state synthesis routes. The process of synthesizing these samples is shown in Fig. \ref{fig:synthesis_procedure}. The constituent elements of the compounds are mixed in certain ratio using a mortar and pestle in an Argon atmosphere glovebox shown in Fig. \ref{fig:synthesis_procedure}(a). Quartz tubes, that have been sealed from one side, are filled with the mixed elemental powders and vacuum sealed using a flame torch as shown in Fig. \ref{fig:synthesis_procedure}(b). The sealed quartz tubes in Fig. \ref{fig:synthesis_procedure}(c) are placed inside a box furnce and heated to a high temperature to allow the powders to fuse together as shown in Fig. \ref{fig:synthesis_procedure}(d). Upon cooling, the desired crystal is removed from the quartz tube and characterized.

\begin{figure}
\centering\includegraphics[width=\columnwidth]{figures/ch3/synthesis_procedure.png} \\
\caption{\label{fig:synthesis_procedure}
The process of making bulk materials using traditional solid state synthesis technique is shown here. The powders mixed inside the glovebox in (a) are transferred to a quartz tube and vacuum sealed in (b). The vacuum sealed ampoule in (c) is placed inside the box furnace in (d) and subjected to a heating profile. (e) shows the final ingot obtained in case of CuMnAs sample.
}
\end{figure}



\subsection{Sensitivity of Fe:As stoichiometry in the synthesis of Fe$_2$As}

% Fe2As and variation with stoichiometry
Fe$_2$As crystals are prepared by mixing Fe and As powders inside the glovebox and heating it up to 600$^\circ$C at 1$^\circ$C/min, holding it for 6 h and then heating it above the melting point to 975$^\circ$C at 1$^\circ$C/min. The sample is held at 975$^\circ$C for 1 h before cooling it down to 900$^\circ$C at 1$^\circ$C/min and held for 1 h. Finally, the sample is allowed to furnace-cool down to the room temperature. The source of the Fe powders seems to have significant effect on the optimum Fe:As starting stoichiometry. As powders were obtained by grinding As chunks (Alfa Aesar, 2-8mm, 99.9999\% (metals basis)) using a mortar and pestle. Fig. \ref{Fe2As_ratio_1} shows the impurity percentage as a function of Fe:As stoichiometry for Fe powders (Alfa Aesar, -200 mesh, 99+\% (metals basis)) with 0.74 $\mu$m in size. Phase pure Fe$_2$As is obtained for Fe:As ratio of 1.95:1. Increasing the Fe content results in Fe impurity and decreasing Fe content results in FeAs impurity.

\begin{figure}
\centering\includegraphics[width=0.5\columnwidth]{figures/ch3/Fe2As_ratio_1.png} \\
\caption{\label{fig:Fe2As_ratio_1}
Impurity percentage as a function of Fe:As starting stoichiometry ratio with 200 mesh size Fe powders for making Fe$_2$As samples.
}
\end{figure}

While using the 200 mesh size Fe powders yielded pure Fe$_2$As, the residual resistivity ratio from the transport measurements indicated Fe$_2$As to be a bad metal. In an attempt to reduce the presence of trace impurities, more pure Fe powders were sourced. Fe powders (Alfa Aesar, 10 $\mu$m, spherical, $>$99.99\% (metals basis)) with 10 $\mu$m in size were used for preparing batches of 1 g of Fe$_2$As crystals. In order to reduce processing errors, before vacuum sealing the quartz tube, a magnet was moved from top to bottom of the outer walls of the tube in a rocking fashion to remove any Fe powders sticking to the inner walls of the tube. Finally, the synthesized Fe$_2$As crystals were crushed into powders and sent to the 11-BM beamline at the Advanced Photon Source in Argonne National Laboratory for synchrotron x-ray diffraction measurements. The results of the 8 samples with different Fe:As ratio is shown in Fig. \ref{fig:Fe2As_ratio_2}(a). The results from this data suggest that Fe:As ratio of 2:1 which is also the stoichiometric ratio, is optimum to produce phase pure samples as shown in Fig. \ref{fig:Fe2As_ratio_2}(b). Similar to earlier results, increasing the Fe:As ratio above 2 precipitates out Fe impurity and decreasing the Fe:As ratio results in FeAs impurity. The region colored in purple in Fig. \ref{fig:Fe2As_ratio_2}(b) contained the opposite of the expected impurity. The purple region on the excess Fe side contained FeAs impurity and on the lower Fe side contained Fe impurity. I attribute this inconsistency to random errors.

\begin{figure}
\centering\includegraphics[width=\columnwidth]{figures/ch3/Fe2As_stoichiometry.png} \\
\caption{\label{fig:Fe2As_ratio_2}
Synchrotron x-ray diffraction data of Fe$_2$As for different ratios of Fe:As from 10 $\mu$m Fe powders is shown in (a) and the corresponding impurity percentage is shown in (b).
}
\end{figure}

\section{Synthesis of compounds in the Cu-Mn-As system}

%Synthesis of CuMnAs compounds

Compounds in the Cu-Mn-As system powders are synthesized by mixing Cu powders (Alfa Aesar, -100+325 mesh, spherical, 99.9\% (metals basis)), Mn powders ground from Mn chips (99.98\% (metals basis)) and As powders ground from chunks (Alfa Aesar, 2-8mm, 99.9999\% (metals basis)) inside the Ar atmosphere and following the same heating procedure as in the case of Fe$_2$As. To synthesize binary compounds Mn$_2$As and Mn$_3$As$_2$, excess Mn has to be added into the mixture. Mn:As ratio of 2.1:1 and 3.1:2 is required to synthesize pure phase Mn$_2$As and Mn$_3$As$_2$, respectively. In case of Mn$_2$As, when Mn and As powders are mixed in stoichiometric ratio, Mn$_3$As$_2$ impurity is formed due to peritectic reaction \cite{Yuzuri1960}. The ratio of Cu:Mn:As powders determines the final product. When Cu:Mn:As ratio is 0.82:1.18:1, the hexagonal polymorph of CuMnAs is stabilized. From literature, when Cu, Mn and As powders are mixed in equal proportions, orthorhombic CuMnAs is formed. However, in our synthesis procedure, we observe a mixture of tetragonal and orthorhombic CuMnAs. The difference in the final product comes from the use of an Alumina crucible inside the quartz tube. I have synthesized tetragonal CuMnAs by substituting equal amounts of Mn with Cu powders. An almost stoichiometric tetragonal CuMnAs has been reported in literature by substituting small amounts of Mn with As \cite{Uhlirova2019}. I also synthesized the near stoichiometric tetragonal CuMnAs by replicating the procedure from literature including the use of Alumina crucible. The heating procedure has a significant impact on the quality of the tetragonal CuMnAs crystals. Tetragonal CuMnAs undergoes a phase transition at around 800$^\circ$C which makes it difficult to synthesize large crystals using traditional solid state synthesis routes. Out of the three elemental powders, Cu is the element that is being directly used in the powder form. Hence, it is prone to oxidation easily. However, Cu powders can be easily reduced using H$_2$ gas flow reaction by heating it to 600$^\circ$C for holding it for 6 hours.

\section{Materials characterization}

\subsection{X-ray diffraction measurements}

% Discuss D8 Cu and Mo, single crystal XRD, Laue and synchrotron

Powder x-ray diffraction measurements were carried out primarily at the Bruker D8 Advance powder x-ray diffractometer with Mo souce using the capillary stage. Since the materials used here contain As and other transition metal atoms which absorb x-rays significantly, thin capillaries of 0.43~mm in diameter were used. In addition to that, the powders were diluted with appropriate amounts of amorphous silica to account for x-ray absorption. Most powder x-ray diffraction measurements on Cu-Mn-As samples were carried out on the Bruker D8 Advance x-ray diffractometer with Cu source using a reflective stage. In these measurements, the samples were not diluted with silica since absorption is not an issue. However, significant sample texturing was observed which had to be taken into account while carrying out Rietveld refinement. Certain samples were also measured at the 11-BM beamline of the Advanced Photon Source in Argonne National Laboratory. The powders were mounted onto 0.7 mm diameter quartz capillaries and vacuum sealed before fitting it inside a Kapton capilliary for measurement. The high energy synchrotron beam wavelength that was used at 11-BM beamline corresponds to 0.4128 \AA. Rietveld refinement of the powder x-ray diffraction data was carried out using \textsc{TOPAS} and \textsc{GSAS-II} \cite{Coelho:jo5037,Toby:aj5212}.

Hexagonal Cu$_{0.82}$Mn$_{1.18}$As crystals were sent to SCS X-ray facility for single crystal x-ray diffraction measurement on a Bruker X8 Apex II diffractometer. Tiny single crystals of the size of around 100 $\mu$m were fractured out from a large ingot and used for the measurement. Alignment of large single crystals of Fe$_2$As and Cu$_{0.82}$Mn$_{1.18}$As for the purpose of aligned SQUID magnetometry, magnetotransport and inelastic neutron scattering measurements were carried out using a Laue System with a Multiwire 2D Detector at both MRL x-ray laboratory as well as at Spallation Neutron Source. The Laue diffractometer was only used to align samples from the symmetry of the Laue pattern. Indexing of the patterns were not carried out using the Northstar software due to issues with the software. The out-of-plane alignment in Cu$_{0.82}$Mn$_{1.18}$As single crystal was also confirmed using the reflective stage of Bruker D8 Advance diffractometer with Mo source.

% add figures for hex CuMnAs alignment

\subsection{SEM/EDS measurements}

% add brief statements about JEOL6060LV and Scios 2

\subsection{Calorimetry measurements}

% Discuss TGA, DSC and DTA

Thermogravimetric analysis (TGA) is a thermal analysis technique where the mass of the sample is tracked over a range of temperatures. It is particularly useful for detecting changes such as sublimation or evaporation of a phase or any kind of absorption/desoprtion processes. Powders or crystals of around 10~mg can be loaded onto an Alumina cup and heated up to 1000$^\circ$C for TGA. All the samples were measured in the Q50 TGA instrument upto above 400$^\circ$C under N$_2$ atmosphere to check if the sample loses its integrity at this temperature range or not. No transitions were observed in any of the samples measured and further analysis was carried out using differential scanning calorimetry (DSC). This measurement technique, although not very useful in this case, is necessary to prevent any accidental coating of the inner chamber during DSC measurements.

DSC is a calorimetry technique where the difference in the heat required to keep the sample at the same temperature as the reference is recorded as a function of temperature. It is a sensitive measurement technique and is useful for detecting magnetic transitions such as the N\'eel temperature or any spin canting transition. DSC measurements for all samples were carried out in the DSC2500 instrument. The sample powders, weighing between 4~mg to 8~mg, were loaded onto Alumina pans and subjected to heat-cool-heat cycles between -180$^\circ$C and 400$^\circ$ at 10$^\circ$C/min.

\begin{figure}
\centering\includegraphics[width=0.8\columnwidth]{figures/ch3/DTA_tubes_combined.jpg} \\
\caption{\label{fig:DTA_tubes_combined}
The flat-bottom bulk of the quartz ampoules is shown in (a) and (b) shows the final powder-containing vacuum sealed ampoules.
}
\end{figure}

Differential thermal analysis (DTA) is similar to DSC except that the temperature difference between the sample and reference is recorded for identical thermal cycles. DTA measurements were carried out on a Shimadzu DTA-50 up to 1200$^\circ$C in some samples at a heating rate of 20$^\circ$C/min under N$_2$ atmosphere. The advantage of using DTA is that it can go up to very high temperatures which is useful for detecting melting point and any other phase transition that is beyond the range of DSC. The presence of As in the samples makes it impossible to use the standard DTA Alumina cups for measurement. To prevent contamination of the room with As vapors, special quartz ampoules were designed and commissioned from the SCS Glass shop. The ampoules were made by cutting 3~mm and 4~mm OD quartz tubes in 85~mm lengths and creating a flat-bottom bulb of 5~mm diameter at one end of the tube as shown in Fig. \ref{fig:DTA_tubes_combined}(a). About 20~mg to 40~mg of sample powders were vacuum sealed in the quartz ampoules and similar amount of Alumina powder was sealed as reference as shown in Fig. \ref{fig:DTA_tubes_combined}(b). The use of quartz ampoules, however, prevents detection of subtle transitions such as magnetic transitions. Hence, DTA can be used in conjunction with DSC to identify all transition temperatures.

\subsection{SQUID magnetometry measurements}

% Discuss squid measurements
SQUID measurements for all the samples were carried out in a Quantum Design MPMS3 (Magnetic Property Measurement System) in the DC mode. The DC moment was tracked as a function of applied field as well as temperature. In magnetic hysteresis measurements, fields up to 10~kOe were applied across six quadrants to account for the initial magnetization curve. Field cooling measurements refer to the application of magnetic field at 400$^\circ$C followed by cooling of the sample. Zero field cooling measurements are carried out by first cooling the sample in the absence of a field, followed by the application of an external field and then, heating the sample. The sample moment is measured at certain intervals across the temperature range. The SQUID measurements have been carried out for the powders of Fe$_2$As, Cu$_{0.82}$Mn$_{1.18}$As, tetragonal CuMnAs and Mn$_3$As$_2$, and the single crystals of Fe$_2$As and Cu$_{0.82}$Mn$_{1.18}$As. In case of powders, 20~mg to 40~mg of powders were filled into a VSM powder sample holder inside the Ar atmosphere glovebox. This capillary was snapped onto a MPMS3 brass half tube sample holder and wrapped with a small amount of insulating tape to keep it fixed. For single crystal measurements, aligned samples were fixed on the MPMS3 Quartz Paddle Sample Holder using GE Varnish and later removed using ethanol.


\subsection{Magnetotransport measurements}

% Discuss PPMS measurements and wire bonding

Magnetotransport measurements for hexagonal Cu$_{0.82}$Mn$_{1.18}$As were carried out in a Quantum Design Physical Property Measurement System Dynacool.

\section{Neutron scattering}

\subsection{Neutron powder diffraction}
% Discuss different instruments and analysis

\subsection{Inelastic neutron scattering}

%Discuss ARCS instrument


%##################chapter 4
\chapter{Magnetic structure refinement from neutron diffraction measurements}

\begin{figure}
\centering\includegraphics[width=0.5\columnwidth]{figures/ch4/C3v.png} \\
\caption{\label{fig:C3v}
Electronic band structure
}
\end{figure}

\begin{figure}
\centering\includegraphics[width=0.5\columnwidth]{figures/ch4/group_multiplication_table_C3v.png} \\
\caption{\label{fig:gmt_C3v}
Electronic band structure
}
\end{figure}

\begin{figure}
\centering\includegraphics[width=0.5\columnwidth]{figures/ch4/C3v_a1_a2.png} \\
\caption{\label{fig:C3v_a1_a2}
Electronic band structure
}
\end{figure}

\begin{figure}
\centering\includegraphics[width=\columnwidth]{figures/ch4/propagation_vector.png} \\
\caption{\label{fig:propagation_vector}
Electronic band structure
}
\end{figure}

\begin{figure}
\centering\includegraphics[width=\columnwidth]{figures/ch4/mag_structures_single_k.png} \\
\caption{\label{fig:mag_structures_single_k}
Electronic band structure
}
\end{figure}

\begin{figure}
\centering\includegraphics[width=0.7\columnwidth]{figures/ch4/star_of_propagation_vector_k.png} \\
\caption{\label{fig:star}
Electronic band structure
}
\end{figure}

\begin{figure}
\centering\includegraphics[width=0.5\columnwidth]{figures/ch4/symmetry_based_analysis.png} \\
\caption{\label{fig:symmetry_based_analysis}
Electronic band structure
}
\end{figure}

\Blindtext[6]


%##################chapter 5
\chapter{Discovery and magnetic frustration of hexagonal Cu$_{0.82}$Mn$_{1.18}$As}

\begin{figure}
\centering\includegraphics[width=\columnwidth]{figures/ch5/phase_diagram_cropped.pdf} \\
\caption{\label{fig:phase_diagram}
Electronic band structure
}
\end{figure}

\begin{figure}
\centering\includegraphics[width=\columnwidth]{figures/ch5/CuMnAs_chemical_structure.png} \\
\caption{\label{fig:chemical_structure}
Electronic band structure
}
\end{figure}

\begin{figure}
\centering\includegraphics[width=\columnwidth]{figures/ch5/h-cumnas_11bm_100k_wand_400k_combine.png} \\
\caption{\label{fig:11BM_WAND}
Electronic band structure
}
\end{figure}

\begin{figure}
\centering\includegraphics[width=\columnwidth]{figures/ch5/dsc-mpms_norm_cropped.pdf} \\
\caption{\label{fig:dsc_mpms}
Electronic band structure
}
\end{figure}

\begin{figure}
\centering\includegraphics[width=\columnwidth]{figures/ch5/WAND_data.png} \\
\caption{\label{fig:WAND_data}
Electronic band structure
}
\end{figure}

\begin{figure}
\centering\includegraphics[width=\columnwidth]{figures/ch5/wand_refinement.png} \\
\caption{\label{fig:wand_refinement}
Electronic band structure
}
\end{figure}

\begin{figure}
\centering\includegraphics[width=\columnwidth]{figures/ch5/CuMnAs_magnetic_structure.png} \\
\caption{\label{fig:CuMnAs_magnetic_structure}
Electronic band structure
}
\end{figure}

\begin{figure}
\centering\includegraphics[width=\columnwidth]{figures/ch5/DOS_CuMnAs.png} \\
\caption{\label{fig:DOS}
Electronic band structure
}
\end{figure}

\begin{figure}
\centering\includegraphics[width=\columnwidth]{figures/ch5/resistivity_data_hall_cropped.pdf} \\
\caption{\label{fig:resistivity}
Electronic band structure
}
\end{figure}

\Blindtext[6]

%##################chapter 6
\chapter{Two step magnetic ordering in monoclinic Mn$_3$As$_2$}

\begin{figure}
\centering\includegraphics[width=\columnwidth]{figures/ch6/monoclinic_Mn3As2_75510.png} \\
\caption{\label{fig:Mn3As2}
Electronic band structure
}
\end{figure}

\begin{figure}
\centering\includegraphics[width=0.5\columnwidth]{figures/ch6/Mn3As2_SEM_image.png} \\
\caption{\label{fig:Mn3As2_SEM}
Electronic band structure
}
\end{figure}

\begin{figure}
\centering\includegraphics[width=\columnwidth]{figures/ch6/FC_ZFC_DSC_Mn3As2_cropped.pdf} \\
\caption{\label{fig:Mn3As2_FC_ZFC_DSC}
Electronic band structure
}
\end{figure}

\begin{figure}
\centering\includegraphics[width=\columnwidth]{figures/ch6/350K_rietveld_diff_temp_NPD_cropped.pdf} \\
\caption{\label{fig:350K}
Electronic band structure
}
\end{figure}

\begin{figure}
\centering\includegraphics[width=\columnwidth]{figures/ch6/250K_mag_structure.png} \\
\caption{\label{fig:250K}
Electronic band structure
}
\end{figure}


\begin{figure}
\centering\includegraphics[width=\columnwidth]{figures/ch6/graph_of_subgraphs_3K.png} \\
\caption{\label{fig:subgraphs}
Electronic band structure
}
\end{figure}

\begin{figure}
\centering\includegraphics[width=\columnwidth]{figures/ch6/3K_mag_structure.png} \\
\caption{\label{fig:3K}
Electronic band structure
}
\end{figure}


\begin{figure}
\centering\includegraphics[width=\columnwidth]{figures/ch6/spin_canting_dps.png} \\
\caption{\label{fig:spin_canting}
Electronic band structure
}
\end{figure}

\begin{figure}
\centering\includegraphics[width=\columnwidth]{figures/ch6/FC_52A_inverse_cropped.pdf} \\
\caption{\label{fig:inv_susceptibility}
Electronic band structure
}
\end{figure}

\Blindtext[6]

%##################chapter 7
\chapter{Spin canting in tetragonal CuMnAs}

\Blindtext[6]

%##################chapter 8
\chapter{Exchange interactions in Fe$_2$As probed by inelastic neutron scattering}

\begin{figure}
\centering\includegraphics[width=0.5\columnwidth]{figures/ch8/Cr2As_INS_magnetic_structure.png} \\
\caption{\label{fig:Cr2As}
Electronic band structure
}
\end{figure}

\begin{figure}
\centering\includegraphics[width=\columnwidth]{figures/ch8/crystal.png} \\
\caption{\label{fig:crystal}
Electronic band structure
}
\end{figure}

\begin{figure}
\centering\includegraphics[width=\columnwidth]{figures/ch8/crystal_array.png} \\
\caption{\label{fig:crystal_array}
Electronic band structure
}
\end{figure}

\begin{figure}
\centering\includegraphics[width=\columnwidth]{figures/ch8/11BM_refinement_elastic_slice.png} \\
\caption{\label{fig:11BM_elastic_slice}
Electronic band structure
}
\end{figure}

\begin{figure}
\centering\includegraphics[width=0.7\columnwidth]{figures/ch8/suppl_misaligned_crystal_kl_slice.png} \\
\caption{\label{fig:misalign_crystal}
Electronic band structure
}
\end{figure}

\begin{figure}
\centering\includegraphics[width=\columnwidth]{figures/ch8/phonon_spectra_magnon_spectra_combined.png} \\
\caption{\label{fig:phonon_magnon_spectra}
Electronic band structure
}
\end{figure}

\begin{figure}
\centering\includegraphics[width=\columnwidth]{figures/ch8/constant_energy_slices.png} \\
\caption{\label{fig:constant_energy_slices}
Electronic band structure
}
\end{figure}

\begin{figure}
\centering\includegraphics[width=\columnwidth]{figures/ch8/suppl_01L_INS_data.png} \\
\caption{\label{fig:01L_spectra}
Electronic band structure
}
\end{figure}

\begin{figure}
\centering\includegraphics[width=\columnwidth]{figures/ch8/suppl_high_energy_data.png} \\
\caption{\label{fig:high_energy_data}
Electronic band structure
}
\end{figure}

\begin{figure}
\centering\includegraphics[width=\columnwidth]{figures/ch8/exp_data_points_0K0dot5.png} \\
\caption{\label{fig:exp_points}
Electronic band structure
}
\end{figure}

\begin{figure}
\centering\includegraphics[width=\columnwidth]{figures/ch8/magnon_spectra_refinement.png} \\
\caption{\label{fig:refinement}
Electronic band structure
}
\end{figure}

\begin{figure}
\centering\includegraphics[width=\columnwidth]{figures/ch8/suppl_simulated_magnon_spectra_3J.png} \\
\caption{\label{fig:3J_fits}
Electronic band structure
}
\end{figure}

\Blindtext[6]

%##################chapter 9
\chapter{Conclusions}

\Blindtext[6]

\end{mainmatter}

% }}}

% {{{ back matter

\begin{backmatter}

\bibliographystyle{ieeetr}
\bibliography{thesis}

\end{backmatter}

%\appendix
%\chapter{My Appendix}

%\Blindtext[6]

% }}}

\end{document}
